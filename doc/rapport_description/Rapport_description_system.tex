\documentclass[12pt,titlepage,french]{article}
\usepackage{babel}
\usepackage{graphicx}
\usepackage[margin=2.5cm]{geometry}
\usepackage{tabularx}
\usepackage[hidelinks]{hyperref}

\usepackage[utf8]{inputenc}
\usepackage[T1]{fontenc}
\pagestyle{plain}

\usepackage{booktabs,makecell,tabu}
\renewcommand\theadfont{\bfseries}

\linespread{1.5}

\begin{document}

\begin{titlepage}
\newcommand{\HRule}{\rule{\linewidth}{0.5mm}}
\center

  \includegraphics[width=0.45\textwidth]{../ressources/img_logos/logo_polytech.png}\\[1cm]
   
  \includegraphics[width=0.45\textwidth]{../ressources/img_logos/logo_taglabs.png}


\HRule \\[0.4cm]
{ \huge \bfseries Description du système \\[0.15cm] }
Classification colorimétrique de nuages de points 3D
\HRule \\[1.5cm]
Ronan Collier,
Mathieu Letrone,
Tri-Thien Truong
\\[1cm]
\today \\ [1cm]
Version 1.0
\end{titlepage}

\tableofcontents % table des matières
\newpage
\listoffigures  % table des figures
\newpage
\section{Contexte et problème}

Taglabs est une jeune entreprise créée il y a deux ans par Yan Koch. L’entreprise s’inscrit dans le domaine de la modélisation 3D d’ouvrages. Ils proposent la modélisation et l’exploitation de nuages de points. Toutefois, ils travaillent surtout en interne sur un logiciel « ScanSap », le but de ce logiciel est d’exploiter les nuages de points 3D avec efficacité et simplicité inégalées.

Voulant continuer leur développement dans ce domaine encore nouveau, l'entreprise cherche maintenant à améliorer leurs outils, afin de compléter l'exploitation des nuages de points. L'ensemble de ces fonctionnalités permettent à leurs clients de pouvoir analyser un environnement en numérique, à un instant précis (qui sera sous la forme d'un scan de nuages de points). Par exemple, une entreprise peut avoir le besoin d'avoir un scan d'une de leur usine, afin d'analyser le positionnement de leurs machines, les potentielles fuites au niveau des tuyaux, etc.

Dans cette optique, l'entreprise souhaite pouvoir intégrer plusieurs fonctionnalité à son système. D'une part, une solution permettant une segmentation basé sur la couleur au sein du ou des nuages de points manipulés. Cette segmentation a pour but de mettre en évidence, isolé les éléments du nuage de point répondant à une plage de couleur demandée. D'autre part, une solution mettant en place une fausse couleur sur un scan (nuage de points) en intensité. Le rôle de la fausse couleur est de mettre en lumière des caractéristiques issu des éléments scannés, et de facilité la compréhension général du nuage en y mettant des couleurs.

l'ensemble des fonctionnalités demandées par le client est une demande de recherche algorithmique sur l'implémentation des solutions. 
\newpage
\section{Solution mise en place : vue globale}

Dans cette partie nous allons présenter l'ensemble des choix que nous avons fait afin de répondre au mieux aux besoins demandés par le client. Et la description de la solution conçue  pour satisfaire lesdit besoins.

\subsection{Choix principaux}

%/!\ 
% Donner vos choix principaux (service, techno, etc.), en expliquant pourquoi vous les avez fait. 
%/!\
Après une phase de recherche et discussion entre les différents partis du projet. Nous nous sommes orientés non plus sur une phase seulement algorithmique mais sur l'implémentaion de nos solutions via un plugin sur un logiciel de manipulation de nuages de points nommés CloudCompare. Ce choix a été fait pour se concentrer davantage sur les algorithmes des solutions que l'implémentation des méthodes de lecture/export de nuages de points.

Le fait de travailler sur cloudCompare à orienté beaucoup de nos choix sur l'environnement de travail et les technologies utilisées. En effet, au vu des nombreuses dépendances, des incations données sur la page git du logiciel, nous avons utilisé Cmake. CMake est système qui permet la vérification des pré-requis nécessaires avant la construction, de déterminer les dépendances afin de planifier la construction adapté à la plateforme.
% j'avoue que je sais pas si c'est vrai nécessaire de le dire

Pour ce qu'il en est du choix du langage, compte tenu que nous développons un plugin sur CloudCompare, il a fallut opter pour le langage qui parviendra au mieux à l'implémentation dudit plugin. Le choix se faisait entre le C++ et le Python, (le tableau compartif est disponible dans le rapport de l'itération une), la décision fut le C++, pour cause, l'ensemble du logiciel est en C++, ce qui permettra de simplifier l'interfaçage et la création du plugin. 
%j'ai l'impression de tourner en rond
Le langage étant rapide et compatible avec de nombreuses bibliothèques. Nous avons décidé de réaliser entièrement le projet en C++, et ainsi éviter de gérer des dépendances uniquement pour la création d'algorithmes en Python.
%Pourquoi c++, et CloudCompare 
\newpage
\subsection{Description de la solution}
%/!\ 
% Décrivez notamment quel est le service offert pour répondre à la problématique (How?), et ce qu’est votre solution concrètement (What?).
%/!\

Nous délivrons à titre de services répondant à la problématique. L'élaboration de recherches dans l'optique non seulement de collecter des informations mais aussi de déterminer quelles pistes suivre ou écarter pour la création des algorithmes des fonctionnalités formulées dans les besoins, qui sont la segmentation basée sur la couleur et la génération de fausses couleurs.
De plus, un apport de valeur en délivrant une expérience utilisateur à travers un plugin encapsulant l'intégralité des algorithmes.
% BOF PEUT MIEUX FAIRE......

\section{Description technologique de la solution}
Notre solution étant un plugin pour CloudCompare, la principale difficultée technique est la compréhension de la structure de ce logiciel et de son API.
En effet, CloudCompare possède une structuration complexe pour quiconque n'est pas habitué à travailler sur des projets C++ d'une telle envergure.
De plus, la documentation possède de nombreuses lacunes, ne présentant pas toutes les classes et n'étant pas à jour. La seule documentation disponible est en ligne et nous n'avons.
Il est donc conseillé de regarder le fonctionnement de l'API en naviguant dans les sources du projet.

\section{Conclusion et perspectives}

\end{document}
