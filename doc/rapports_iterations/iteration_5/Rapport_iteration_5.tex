\documentclass[12pt,titlepage,french]{article}
\usepackage{babel}
\usepackage{graphicx}
\usepackage[margin=2.5cm]{geometry}

\usepackage[hidelinks]{hyperref}
\usepackage{tabularx}
\usepackage{float}
\usepackage[utf8]{inputenc}
\usepackage[T1]{fontenc}
\pagestyle{plain}

\usepackage{booktabs,makecell,tabu}
\usepackage{comment}
\renewcommand\theadfont{\bfseries}

\linespread{1.5}

\newcounter{firstbib}

\begin{document}
%\renewcommand{\thesection}{\arabic{section}} % utilisé pour spécifier la numérotation des sections

\begin{titlepage}
\newcommand{\HRule}{\rule{\linewidth}{0.5mm}}
\center

  \includegraphics[width=0.45\textwidth]{../../ressources/img_logos/logo_polytech.png}\\[1cm]

  \includegraphics[width=0.45\textwidth]{../../ressources/img_logos/logo_taglabs.png}


\HRule \\[0.4cm]
{ \huge \bfseries Rapport itération 5\\[0.15cm] }
Classification colorimétrique de nuages de points 3D\\
Version 1.0\\
Le \today \\
\HRule \\[1.5cm]
Ronan Collier,
Mathieu Letrone,
Tri-Thien Truong
\\[1cm]
\end{titlepage}

\tableofcontents % table des matières
\newpage
\listoffigures  % table des figures
\newpage

\section{Rappel des objectifs de l'itération}
Suite à l'itération où nous avions amélioré notre solution, nous avons voulu élargir nos types de filtrages, en utilisant d'autres espaces colorimétriques. De plus, nous avions encore des tâches en cours, que nous devions avancer/finaliser.

Durant l'itération précédente, nous avons pu implémenter un nouveau filtre, en utilisant l'espace colorimétrique HSV. Il nous fallait maintenant avoir des retours clients pour que l'on puisse corriger les différents bugs. De plus, des tâches majeurs étaient encore en cours.

Les tâches que nous nous sommes fixées sont les suivantes :

\begin{itemize}
    \item Restreindre l'utilisateur lors de la saisie du filtrage RGB (borne inférieure et supérieure)
    \item Amélioration des plages du filtrage HSV + test de performance
    \item Résoudre les divers problèmes techniques qui empêchent le fonctionnement de l'algorithme de segmentation
    \item Résoudre problèmes et débuggage du toon mapping
    \item Comparer les performances avec un autre algorithme de toon mapping
\end{itemize}

Nous avions aussi une tâche qui s'est ajoutée pendant l'itération, qui était d'exporter le plugin pour pouvoir l'utiliser sur toutes les machines, en utilisant CloudCompare. Nous avions aussi eu des retours utilisateurs, afin d'améliorer notre plugin.

\section{Production / réalisation durant l'itération}

Nous développerons ici chaque objectif que nous nous sommes fixé pour cette itération.


\subsection{Exportation du plugin}

Dans le but d'avoir des retours de notre client, nous avons voulu exporter notre plugin pour qu'il soit utilisable avec le logiciel CloudCompare, sur une machine quelconque. \newline

Pour ce faire, le principe est de récupérer le fichier ColorimetricSegmenter.dll que nous obtenons lorsque nous compilons le projet, et de l'intégrer dans le dossier "plugins" dans le repertoire d'installation de CloudCompare. \newline

Nous avons rencontrés des difficultés lors de cette tâche. En effet, nous avons vu qu'il y avait des problèmes de versions des dépendances, qui ne faisaient pas fonctionner notre plugin sur toutes les machines. \newline

En effet, lorsque nous compilions notre projet, nous utilisions Visual Studio 2019. En utillisant cette version, la version du windows SDK était aussi modifié, et c'était la source du problème de compatibilité entre nos machines, qui avaient une version de windows plus récente que les machines de notre client, ou encore M. Daniel Girardeau-Montaut, le créateur de CloudCompare.

\begin{figure}[H]
 \caption{\label{} Notre version du SDK Windows, avec Visual Studio 2019}
 \begin{center}
 \includegraphics[width=1\textwidth]{./img/vs2019.PNG}
  \end{center}
\end{figure}


\begin{figure}[H]
 \caption{\label{} Version du SDK Windows correcte, avec Visual Studio 2017}
 \begin{center}
 \includegraphics[width=1\textwidth]{./img/vs2017.PNG}
  \end{center}
\end{figure}

Grâce à cette solution, le client a pu tester notre plugin de son côté, et a pu nous faire des retours concrets sur l'utilisation de notre plugin.

\subsection{Restreindre l'utilisateur lors de la saisie du filtrage RGB}

Lorsque nous avons décidé cette tâche, nous voulions améliorer le filtrage RGB côté utilisateur, afin de mieux choisir les bornes minimum et maximum. \newline

En effet, l'utilisation de l'espace colorimétrique RGB limite les bornes que l'on peut utiliser, afin de sélectionner les points que l'on veut. Pour que ce filtrage fonctionne correctement, nous sommes obligé de sélectionner deux points qui ont la même gamme de couleur. Il faut ensuite choisir le point le plus "sombre", c'est-à-dire avec les valeurs RGB les plus faibles, puis le point le plus clair. \newline

Plusieurs problèmes peuvent se poser. Par exemple, si l'utilisateur décide de prendre des points dans des gammes de couleurs différentes, ce qui donne un résultat non satisfaisant. Il peut aussi avoir un problème lorsque l'utilisateur inverse la sélection des points, ce qui inverse les bornes. Il n'y aurait donc aucun point de sélectionné. \newline

Une solution peut consister à inverser les valeurs, si l'utilisateur se trompe. Pour cela, il faut être sûr que l'utilisateur a bien choisi deux points de la même gamme de couleur, puis vérifier que les trois composantes du deuxième point sont inférieurs au premier point, afin de les inverser. \newline

Une question reste en suspend, qui est de savoir si nous réalisons le traitement directement sur l'interface (l'utilisateur verra donc directement le changement des bornes), ou si le traitement se fera au niveau en pré-traitement du parcours de points, et l'utilisateur aura juste à choisir ses deux points.

\subsection{Amélioration des plages du filtrage HSV}



\subsection{Algorithme de segmentation}

\subsection{Résoudre problèmes et débuggage du toon mapping}

\subsection{Comparer les performances avec un autre algorithme de toon mapping}

\section{Risques éliminés durant l'itération}


\section{Feedback}

\section{Commentaires sur l'itération}

Cette section va présenter nos ressentis sur notre itération. Cela peut correspondre à la façon dont nous avons pu gérer la charge de travail que nous avions prévu en début d'itération, des potentiels imprévus, points positifs/négatifs, et autres.

\subsection{Commentaires sur l'itération de façon générale}

Tout d'abord, nous avons dû replanifier notre itération, notamment à cause du coronavirus, qui a provoqué l'annulation du projet ICreate. Nous avions prévu que cette itération allait être compliqué à cause de ce projet. Avec son annulation, il nous fallait revoir notre planning, pour réaliser nos tâches sur cette itération, et la prochaine. \newline

De plus, un membre du projet a été victime de ce virus, et était dans l'incapacité de travailler. Il va maintenant mieux, et va pouvoir reprendre ses tâches. \newline

Concernant les différentes tâches, nous avons eu beaucoup de mal à cerner le problème de compatibilité de notre plugin avec les machines de notre client. En effet, comme notre plugin fonctionnait sur nos machines, le seul moyen pour trouver le problème était de le faire tester directement avec notre client. \newline

Comme notre moyen de discussion était via email, le délai entre questions et réponses, puis les tests avec les retours, était assez long. Heureusement, nous avons pu trouver une solution grâce au créateur du plugin M. Daniel Girardeau-Montaut, et nous le remercions grandement pour le temps qu'il a pu nous accorder afin de résoudre notre problème.

\subsection{Commentaires sur les méthodes de travail/changements de méthode}


\section{Objectifs de la prochaine itération}


\section{Résumé}
\subsection{Tâches principales réalisées dans l'itération}

\subsection{Tâches principales à réaliser pour la prochaine itération}

\begin{thebibliography}{3}

\end{thebibliography}
\end{document}
