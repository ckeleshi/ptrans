\documentclass[12pt,titlepage,french]{article}
\usepackage{babel}
\usepackage{graphicx}
\usepackage[margin=2.5cm]{geometry}

\usepackage[hidelinks]{hyperref}
\usepackage{tabularx}
\usepackage[utf8]{inputenc}
\usepackage[T1]{fontenc}
\pagestyle{plain}

\usepackage{booktabs,makecell,tabu}
\renewcommand\theadfont{\bfseries}

\linespread{1.5}

\begin{document}
%\renewcommand{\thesection}{\arabic{section}} % utilisé pour spécifier la numérotation des sections

\begin{titlepage}
\newcommand{\HRule}{\rule{\linewidth}{0.5mm}}
\center

  \includegraphics[width=0.45\textwidth]{../../ressources/img_logos/logo_polytech.png}\\[1cm]
   
  \includegraphics[width=0.45\textwidth]{../../ressources/img_logos/logo_taglabs.png}


\HRule \\[0.4cm]
{ \huge \bfseries Rapport itération 1\\[0.15cm] }
Classification colorimétrique de nuages de points 3D\\
Version 1.0\\
Le \today \\
\HRule \\[1.5cm]
Ronan Collier,
Mathieu Letrone,
Tri-Thien Truong
\\[1cm]
\end{titlepage}

\tableofcontents % table des matières
\newpage

\section{Rappel des objectifs de l'itération}
La présente itération avait pour objectifs principaux l'étude des techniques qui vont être utilisées pour les itérations suivantes.
Il s'agit d'un travail principalement de documentation.

\section{Production / réalisation durant l'itération}
Nous avons produit un comparatif des bibliothèques C++ et Python permettant la manipulation de nuage de points et la segmentation.
Nous avons également développé des scripts Python courts permettant de nous familiariser avec les nuages de points.

\section{Risques éliminés durant l'itération}
L'itération n'a pas permise l'élimination de risques. En effet, nous n'avons que peux expérimenté sur les données fournies.

\section{Commentaires sur l'itération}

\subsection{Commentaires sur l'itération de façon générale}
L'organisation de l'itération a été difficile étant donné la présence de nombreux examens et rendus de projets dans les semaines.

\subsection{Commentaires sur les méthodes de travail/changements de méthode}
Nous nous sommes répartis les différentes recherches à réaliser. 

\section{Trois principaux risques restants}
\begin{itemize}
  \item Incapacité à distinguer les éléments de façon efficace dus à des artéfacts ou manques d'informations sur une zone du nuage de points
  \item Impact trop important du mouchetage sur la qualité de la segmentation.
  \item Des points résiduels qui restent dans le nuage de points après filtrage (notamment à cause des artéfacts).  
\end{itemize}

\section{Objectifs de la prochaine itération}
Développement des fonctionnalités principales de la solution à partir des recherches éffectuées.

\section{Résumé}

\subsection{Tâches principales réalisées dans l'itération}
\noindent\begin{tabu} to \textwidth {p{0.15\textwidth}X[c2]X[c3]X[c3]}\toprule
  \thead{Tâche}&\thead{Responsable}&\thead{Statut}&\thead{Commentaire}\\\toprule
Recherche sur les méthodes de fausse couleurs
& Ronan
& Achevé
& Peux d'informations décrivant la création d'une palette de couleurs\\\midrule
gdhsh
& sh
& hsh
& hsh\\\bottomrule \\
\end{tabu}

\subsection{Tâches principales à réaliser pour la prochaine itération}
\begin{itemize}
  \item Développement de la sélection de points selon la couleur
  \item Développement de suppression de points
  \item Développement du lien entre sélection et suppression
\end{itemize}
\end{document}

