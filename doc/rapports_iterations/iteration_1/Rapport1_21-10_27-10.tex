\documentclass[12pt,titlepage,french]{article}
\usepackage{babel}
\usepackage{graphicx}
\usepackage[margin=2.5cm]{geometry}

\usepackage[hidelinks]{hyperref}
\usepackage{tabularx}
\usepackage[utf8]{inputenc}
\usepackage[T1]{fontenc}
\pagestyle{plain}

\usepackage{booktabs,makecell,tabu}
\renewcommand\theadfont{\bfseries}

\linespread{1.5}

\begin{document}
%\renewcommand{\thesection}{\arabic{section}} % utilisé pour spécifier la numérotation des sections

\begin{titlepage}
\newcommand{\HRule}{\rule{\linewidth}{0.5mm}}
\center

  \includegraphics[width=0.45\textwidth]{../../ressources/img_logos/logo_polytech.png}\\[1cm]
   
  \includegraphics[width=0.45\textwidth]{../../ressources/img_logos/logo_taglabs.png}


\HRule \\[0.4cm]
{ \huge \bfseries Rapport itération 1\\[0.15cm] }
Classification colorimétrique de nuages de points 3D\\
Version 1.0\\
Le \today \\
\HRule \\[1.5cm]
Ronan Collier,
Mathieu Letrone,
Tri-Thien Truong
\\[1cm]
\end{titlepage}

\tableofcontents % table des matières
\newpage

\section{Rappel des objectifs de l'itération}


\section{Production / réalisation durant l'itération}


\section{Risques éliminés durant l'itération}


\section{Commentaires sur l'itération}

\subsection{Commentaires sur l'itération de façon générale}

\subsection{Commentaires sur les méthodes de travail/changements de méthode}


\section{Trois principaux risques restants}


\section{Objectifs de la prochaine itération}

\section{Résumé}

\subsection{Tâches principales réalisées dans l'itération}

\subsection{Tâches principales à réaliser pour la prochaine itération}

\end{document}

