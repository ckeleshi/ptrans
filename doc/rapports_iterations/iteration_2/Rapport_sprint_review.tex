\documentclass[12pt,titlepage,french]{article}
\usepackage{babel}
\usepackage{graphicx}
\usepackage[margin=2.5cm]{geometry}

\usepackage[hidelinks]{hyperref}
\usepackage{tabularx}
\usepackage[utf8]{inputenc}
\usepackage[T1]{fontenc}
\pagestyle{plain}

\usepackage{booktabs,makecell,tabu}
\renewcommand\theadfont{\bfseries}

\linespread{1.5}

\newcounter{firstbib}

\begin{document}
%\renewcommand{\thesection}{\arabic{section}} % utilisé pour spécifier la numérotation des sections

\begin{titlepage}
\newcommand{\HRule}{\rule{\linewidth}{0.5mm}}
\center

  \includegraphics[width=0.45\textwidth]{../../ressources/img_logos/logo_polytech.png}\\[1cm]

  \includegraphics[width=0.45\textwidth]{../../ressources/img_logos/logo_taglabs.png}


\HRule \\[0.4cm]
{ \huge \bfseries Rapport de Sprint review\\[0.15cm] }
Classification colorimétrique de nuages de points 3D\\
Version 1.0\\
Le \today \\
\HRule \\[1.5cm]
Ronan Collier,
Mathieu Letrone,
Tri-Thien Truong
\\[1cm]
\end{titlepage}

\tableofcontents % table des matières
\newpage
\listoffigures  % table des figures
\newpage

\section{Rappel de l'objectif de l'itération}

Suite à notre première itération qui était dédié à de la recherche permettant de faire nos choix d'implémentations, cette itération avait pour objectif de commencer à développer notre solution. En effet, le principal objectif était de développer nos premières méthodes de filtrage d'un nuage de points, et de les intégrer au logiciel CloudCompare sous forme de plugin.

Pour cela, il nous fallait, d'une part, développer nos traitements sur les nuages de points via la bibliothèque PCL. D'autre part, nous devions installer l'environnement permettant le développement du plugin CloudCompare, c'est-à-dire, pouvoir utiliser le projet git CloudCompare et d'y intégrer nos filtres.

\section{Rappel du sprint backlog}

Les User stories sélectionnés pour atteindre l'objectif de l'itération, sont les suivants :

\begin{itemize}
  \item US1 : Lire un fichier de données contenant un nuage de points
  \item US2 : Isoler un élément dans le nuage de points, selon sa plage de couleur
  \item US3 : Exporter le nuage de points
\end{itemize}

Comme nous avons fait un changement par rapport à nos choix de conception du projet lors de la première itération, les User stories sont aussi amenés à être modifié. Lors de la réalisation du cahier des charges, nous pensions qu'il fallait développer nous même la lecture d'un fichier avec un nuage de points, et l'exportation du nuage. Suite à la première itération, nous avons décidé de réaliser un plugin sur le logiciel CloudCompare, donc ces tâches étaient déjà effectuées directement sur ce dernier. \\
Ces deux User stories ont été modifié par : 
\begin{itemize}
  \item US1 : Intégrer un plugin CloudCompare
  \item US2 : Isoler un élément dans le nuage de points, selon sa plage de couleur
  \item US3 : Créer des sous-scans suite au filtrage\\
\end{itemize}

Les modifications des User stories impliquent donc des modifications par rapport aux tests d'acceptations. Nous les avons donc redéfinies pour qu'ils correspondent mieux par rapport à la solution voulue par le client.

\textbf{\og US1 : Intégrer un plugin CloudCompare :\fg{}}

En tant qu'utilisateur du logiciel CloudCompare, je souhaite voir dans la liste des plugins, un nouveau filtre personnalisé afin d'y ajouter un nouvel algorithme de segmentation.

\textbf{Tests d'acceptation de l'US1 :}

\begin{enumerate}
    \item \textbf{Scénario 1}

Étant donné que je suis sur le logiciel CloudCompare\\
Quand j'affiche la liste des plugins\\
Et que j'affiche les filtres en passant ma souris sur "PCL wrapper" \\
Alors un nouveau bouton du filtre personnalisé apparaît désactivé

    \item \textbf{Scénario 2}

Étant donné que je suis sur le logiciel CloudCompare\\
Quand je charge un fichier de nuage de points\\
Alors le scan s'affiche dans la liste des scans\\
Quand je clique sur le scan\\
Et que j'affiche la liste des filtres via les plugins\\
Alors le bouton du filtre personnalisé apparaît actif\\

\end{enumerate}

\textbf{\og US2 : Isoler un élément dans le nuage de points, selon sa plage de couleur\fg{}}

En tant qu'utilisateur du logiciel, je souhaite isoler un élément dans le nuage de points afin d'afficher à l'écran uniquement l'élément voulu selon sa couleur.

\textbf{Tests d'acceptation :}
\begin{enumerate}

    \item \textbf{Scénario 1}

Étant donné que j'ai affiché un nuage de points via un fichier\\
Quand je clique sur l'option pour isoler un élément\\
Alors une interface avec un color picker apparaît\\
Quand je sélectionne une plage de couleur\\
Et que je clique sur le bouton de validation\\
Alors l'interface disparaît\\
Et le nuage de points n'affiche uniquement que les points respectant la plage d'intensité de couleur.

    \item \textbf{Scénario 2}

Étant donné que j'ai affiché un nuage de points via un fichier\\
Quand je clique sur l'option pour isoler un élément\\
Alors une interface avec un color picker apparaît\\
Quand je sélectionne une plage de couleur\\
Et que je clique sur le bouton d'annulation\\
Alors l'interface disparaît\\
\end{enumerate}

\textbf{\og US3 : Créer des sous-scans suite au filtrage\fg{}}

En tant qu'utilisateur du logiciel, je souhaite créer des sous-scans après filtrage, afin de séparer les points externes et les points internes à la sélection.

\textbf{Tests d'acceptation :}
\begin{enumerate}

    \item \textbf{Scénario 1}

Étant donné que j'ai affiché un nuage de points via un fichier\\
Quand je clique sur le bouton pour filtrer mon nuage de points\\
Alors une interface de paramètres apparaît\\
Quand je valide les paramètres\\
Alors l'interface disparaît\\
Alors des sous-scans s'affichent dans la liste des scans
\end{enumerate}

\section{Problèmes rencontrés et imprévus}

\section{Items réalisés / terminés}

\begin{itemize}
  \item US1 : ...
\end{itemize}

\section{Démonstration (déroulement, USs testés, cas d'utilisation)}

\section{Retours du client (feedback)}

\section{Vision pour la suite}


\noindent\begin{tabularx}{\textwidth}{|X|X|X|}
    \hline
    \textbf{Signature du chef de projet du groupe étudiant :} & \textbf{Signature du tuteur académique :} & \textbf{Signature du tuteur industriel}\\
    \hline
   \rule{0pt}{3cm} &
    &\\
    \hline
\end{tabularx}

\end{document}
