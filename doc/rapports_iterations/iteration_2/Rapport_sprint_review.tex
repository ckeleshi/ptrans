\documentclass[12pt,titlepage,french]{article}
\usepackage{babel}
\usepackage{graphicx}
\usepackage[margin=2.5cm]{geometry}

\usepackage[hidelinks]{hyperref}
\usepackage{tabularx}
\usepackage[utf8]{inputenc}
\usepackage[T1]{fontenc}
\pagestyle{plain}

\usepackage{booktabs,makecell,tabu}
\renewcommand\theadfont{\bfseries}

\linespread{1.5}

\newcounter{firstbib}

\begin{document}
%\renewcommand{\thesection}{\arabic{section}} % utilisé pour spécifier la numérotation des sections

\begin{titlepage}
\newcommand{\HRule}{\rule{\linewidth}{0.5mm}}
\center

  \includegraphics[width=0.45\textwidth]{../../ressources/img_logos/logo_polytech.png}\\[1cm]

  \includegraphics[width=0.45\textwidth]{../../ressources/img_logos/logo_taglabs.png}


\HRule \\[0.4cm]
{ \huge \bfseries Rapport de Sprint review\\[0.15cm] }
Classification colorimétrique de nuages de points 3D\\
Version 1.0\\
Le \today \\
\HRule \\[1.5cm]
Ronan Collier,
Mathieu Letrone,
Tri-Thien Truong
\\[1cm]
\end{titlepage}

\tableofcontents % table des matières
\newpage
\listoffigures  % table des figures
\newpage

\section{Rappel de l'objectif de l'itération}

\section{Rappel du sprint backlog}

\begin{itemize}
  \item US1 : ...
\end{itemize}

\section{Problèmes rencontrés et imprévus}

\section{Items réalisés / terminés}

\begin{itemize}
  \item US1 : ...
\end{itemize}

\section{Démonstration (déroulement, USs testés, cas d'utilisation)}

\section{Retours du client (feedback)}

\section{Vision pour la suite}


\end{document}
