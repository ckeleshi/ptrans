\documentclass[12pt,titlepage,french]{article}
\usepackage{babel}
\usepackage{graphicx}
\usepackage[margin=2.5cm]{geometry}

\usepackage[hidelinks]{hyperref}

\usepackage[utf8]{inputenc}
\usepackage[T1]{fontenc}
\pagestyle{plain}

\usepackage{booktabs,makecell,tabu}
\renewcommand\theadfont{\bfseries}

\linespread{1.5}

\begin{document}
%\renewcommand{\thesection}{\arabic{section}} % utilisé pour spécifier la numérotation des sections

\begin{titlepage}
\newcommand{\HRule}{\rule{\linewidth}{0.5mm}}
\center

  \includegraphics[width=0.45\textwidth]{./image2.png}\\[1cm]
   
  \includegraphics[width=0.45\textwidth]{./image1.png}


\HRule \\[0.4cm]
{ \huge \bfseries Cahier des charges \\[0.15cm] }
Classification colorimétrique de nuages de points 3D
\HRule \\[1.5cm]
Ronan Collier,
Mathieu Letrone,
Tri-Thien Truong
\\[1cm]
\today \\ [1cm]
Version 1.1
\end{titlepage}

\section{Introduction}

Nous tenons avant tout à remercier l'entreprise TagLabs pour avoir participer à la rédaction de ce cahier des charges.

\section{Contexte general du projet dans l'entreprise}

Taglabs est une jeune entreprise créée il y a deux ans par Yan Koch. L’entreprise s’inscrit dans le domaine de la modélisation 3D d’ouvrages. Ils proposent la modélisation et l’exploitation de nuages de points. Mais, ils travaillent surtout en interne sur un logiciel « ScanSap », le but de ce logiciel est d’exploiter les nuages de points 3D avec efficacité et simplicité inégalées.

Voulant continuer leur développement dans ce domaine encore nouveau, l'entreprise cherche maintenant à améliorer leurs outils, afin de compléter l'exploitation des nuages de points. L'ensemble de ces fonctionnalités permettent à leurs clients de pouvoir analyser un environnement en numérique, à un instant précis (qui sera sous la forme d'un scan de nuages de points). Par exemple, une entreprise peut avoir le besoin d'avoir un scan d'une de leur usine, afin d'analyser le positionnement de leurs machines, les potentielles fuites au niveau des tuyaux, etc.

L'équipe qui sera à la charge de ce projet est composé de trois étudiants en informatique à Polytech Nantes. Dans le cadre de la quatrième année dans la formation d'ingénieur informatique, nous devons réaliser un projet transversal avec une entreprise.

\subsection*{MOA}

\noindent\begin{tabu} to \textwidth {X[c]X[c]X[c]X[c]}\toprule
   \thead{Qui}&\thead{Rôle}&        \thead{Mail}&\thead{Mobile}\\\toprule
      Yan Koch&   Président&  yankoch@taglabs.fr&    0660239733\\\midrule
Robin Kervadec&   Ingénieur&rkervadec@taglabs.fr&    0619656021\\\bottomrule
\end{tabu}

\subsection*{MOE}

\noindent\begin{tabu} to \textwidth {X[c2]X[c]X[c3]X[c]}\toprule
     \thead{Qui}&\thead{Rôle}&                       \thead{Mail}&\thead{Mobile}\\\toprule
Tri-thien Truong& Développeur&tri-thien.truong@etu.univ-nantes.fr&    0631193663\\\midrule
   Ronan Collier& Développeur&   ronan.collier@etu.univ-nantes.fr&    0666847162\\\midrule
 Mathieu Letrone& Développeur& mathieu.letrone@etu.univ-nantes.fr&    0789662916\\\bottomrule
\end{tabu}

\subsection*{Tuteur}

\noindent\begin{tabu} to \textwidth {X[c2]X[c2]X[c3]X[c]}\toprule
     \thead{Qui}&\thead{Rôle}&                       \thead{Mail}&\thead{Mobile}\\\toprule
Nicolas Normand& Département informatique&Nicolas.Normand@univ-nantes.fr&    0240683207\\\bottomrule
\end{tabu}

\section{Modèle du domaine / vocabulaire}

La solution devra permettre d’isoler, classifier un élément dans le nuage de points pour une meilleure visibilité et compréhension. Pour cela, la classification se basera selon la plage de couleur
\begin{itemize}
	\item  s'il est déjà en couleur: on applique le filtre de plage colorimétrique pour isoler les éléments (exemple: isoler les tubes proches du blanc).\par

	\item  s'il est en intensité de gris, on passe d'abord par de la fausse couleur, étape qui permet à l'utilisateur de mieux percevoir le nuage, puis on applique le filtre par plage colorimétrique.\par

\end{itemize}

Vocabulaire :

TODO

\section{Besoins fonctionnels}

\noindent\begin{tabu} to \textwidth {X[c2]X[c]X[c]X[c3]}\toprule
\thead{Fonctions}&
\thead{Critères}&                       
\thead{Niveaux}&
\thead{Flexibilité}\\\toprule
Lire un fichier de données contenant un nuage de points. 
& Type de fichier 
& Fichier au format ASCII
& L'extension du fichier ainsi que l'ordre des points et des données de ces points est laissé libre. \\\midrule
Isoler un élément dans un nuage de points donné, selon sa plage de couleur 
& Correspondance entre la plage de couleur désirée et les points isolés.
Sélection des points appartenant à l'objet ciblé mais de teintes différentes (ombrages, réflexion, ...).
& blablabla
& blablabla\\\midrule
Faire apparaître des couleurs sur des nuages de points en intensité de gris
& Distinction des différentes nuances de gris
& blablabla
& blablabla\\\bottomrule

\end{tabu}


\section{Besoins non fonctionnels}

\noindent\begin{tabu} to \textwidth {X[c]X[c3]}\toprule
     \thead{Fonctions}&\thead{Commentaires}\\\toprule
Traitement des problèmes de "mouchetage"
& Dans un premier temps, le mouchetage ne sera pas pris en compte dans le traitement du nuage de points.
\\\midrule
Sélection de la plage de couleurs à extraire 
& Le programme fournit une interface graphique comportant un "color-picker" permettant à l'utilisateur d'effectuer la sélection.\\\midrule
Performances de l'ordinateur
& A cause des nuages de points, il est nécessaire d'avoir des ordinateurs un minimum performants. Cela veut donc impliquer le fait que le programme développé nécessitera aussi un ordinateur performant.\\\midrule
Fichiers en entrée
& L'utilisation du programme devra vérifier la conformité des scans que l'on nous fournira.\\\midrule
Language utilisé
& Il n'y a aucun language requis par le client, nous utiliserons Python, un langage maitrisé par toute l'équipe de développement et adapté à la manipulation de données telles que des nuages de points\\\bottomrule
\end{tabu}

\section{Evolution potentielle des besoins}

Intégration des algorithmes développés dans le logiciel de l'entreprise.


\section{Limites du projet}

La correction des erreurs et artefacts des fichiers fournis en entrée du programme ne font pas partie du projet.
De même que la visualisation du nuage de points directement depuis le programme. Cette fonctionalité est assurée par le logiciel de l'entreprise, plus performant. Cette fonctionnalité peut être présente mais uniquement à des fins de test.

\section{Risques}

Incapacité à distinguer les éléments de façon efficace dû à des artefacts ou manques d'informations sur une zone du nuage de points.\\
Impact trop important du mouchetage sur la qualité de la segmentation.\\
Panne du matériel de travail.

\section{Plannings}

TODO

\section{Gestion de projet}

TODO

\section{Connaissances utiles}

TODO
\end{document}
